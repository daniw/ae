\section{Lösung}
Die Leistung wird mit Spulen L, welche als Tiefpass wirken in die Leitung 
eingekoppelt. Die Daten werden mit Kondensatoren als Hochpass in die Leitung 
eingekoppelt. 
\begin{figure}[h!]
    \centering
    \begin{circuitikz}
        \draw
            % Power path
            ( 0,6) to[L=L, -*] (3,6) -- (12,6) to[L=L, *-] (15,6)
            ( 0,4) to[L=L]     (3,4) -- (12,4) to[L=L]     (15,4)
            ( 0,6) to[V=24<\volt>, o-o] (0,4)
            (15,6) to[R=$R_L$, o-o] (15,4)
            % Data path
            ( 0,2) -| ( 3,2) to[C=C]     (3,4) -- (3,6)
            ( 0,0) -| ( 5,2) to[C=C, -*] (5,4)
            (15,2) -| (12,2) to[C=C]     (12,4) -- (12,6)
            (15,0) -| (10,2) to[C=C, -*] (10,4)
            ( 2,2) to[R=$R_{CAN}$, *-*] ( 2,0)
            (13,2) to[R=$R_{CAN}$, *-*] (13,0)
        ;
    \end{circuitikz}
    \caption{Prinzipschaltbild der Lösung}
    \label{fig:sol}
\end{figure}
Dabei müssen folgende Impedanzen beachtet werden: 
\paragraph{Power}
\[ X_L = j \cdot \omega \cdot L \]
\[ X_{L_{P}}   = j \cdot \omega_{P_{max}}   \cdot L << R_{L} \]
\[ X_{L_{CAN}} = j \cdot \omega_{CAN_{min}} \cdot L >> R_{L} \]
\paragraph{CAN}
\[ X_C = \frac{1}{j \cdot \omega \cdot C} \]
\[ X_{C_{P}}   = \frac{1}{j \cdot \omega_{P_{max}}   \cdot C} >> R_{CAN} \]
\[ X_{C_{CAN}} = \frac{1}{j \cdot \omega_{CAN_{min}} \cdot C} << R_{CAN} \]
