\section{Mischer}
Als Mischer wird ein Mark Space Modulator \cite{???} verwendet. Dieser 
basiert darauf, ein Pulsweitenmoduliertes Signal zu generieren. Die Amplitude 
wird dabei vom Signal A definiert, Das Puls-Pausenverhältnis vom Signal B. 
Wird dieses Signal dann gefiltert, entsteht dabei das Produkt der beiden 
Eingangssignale. Wichtig dafür ist jedoch, dass die Frequenz der PWM über 
der Frequenz der Nutzsignale liegt. Ansonsten kann diese nicht mehr 
weggefiltert werden. 
